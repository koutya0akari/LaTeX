% 通常 platex + dvipdfmx で動作している
% (u)platex は DVI 出力エンジン, 対して pdfTeX, LuaTeX, XeTeX PDF 出力エンジン
% DVI 出力エンジンを使用する場合, ドライバ指定をする必要がある
%\documentclass[dvipdfmx]{bxjsreport}
\documentclass[dvipdfmx]{jsarticle}
% \documentclass[dvipdfmx]{bxjsbook}
% dvipdfmx でドライバ指定している理由は graphicx,xcolor などがドライバ依存のパッケージであるため
%%%%%%%%%%%%%%%%%%%%%%%%%%%%%%%%%%%%%%%%%%%%%%%%%%%%%%%%%%%%%%%%%%%%%%%%%%%%%%%%%%%
%% \usepackage{graphicx}
% png や jpeg といった図を pdf に挿入できるようにする
% ※ 注意 : tikzより前に読み込む必要あり
%\usepackage{xcolor} % マニュアル : コマンドラインから texdoc xcolor を実行
% xoclor : 文字などの書式に色をつけることができる
% デフォルト19色 : red,green,blue,cyan,magenta,yellow,black,gray,white,darkgray,lightgray,brown,lime,olive,orange,pink,purple,teal,violet
% ※ 注意 : graphicx,xcolor はどちらも dvipdfmx のドライバ指定が必須
% ※ 注意 : オプションを使用する場合, tikzより前に読み込む必要あり
%%%%%%%%%%%%%%%%%%%%%%%%%%%%%%%%%%%%%%%%%%%%%%%%%%%%%%%%%%%%%%%%%%%%%%%%%%%%%%%%%%%
%% 基本 %%

\usepackage{amsmath,mathtools}
% 数式コマンド equation, align などを使えるようにする
\usepackage{amsthm}
% 定理環境 theoremstyle 制作
\usepackage{amssymb}
% 基本的な数学記号などの導入
\usepackage{latexsym}
% 数式で使える記号を増やす
%%%%%%%%%%%%%%%%%%%%%%%%%%%%%%%%%%%%%%%%%%%%%%%%%%%%%%%%%%%%%%%%%%%%%%%%%%%%%%%%%%%
%% tikz (てぃくす)関係 %% % https://texwiki.texjp.org/?TikZ

\usepackage{tikz}
% \begin{tikzpicture} hoge \end{tikzpicture} により図式を書けるようにする
% beamer の基礎エンジンでもある
% ※ 注意 : graphicx,xcolor 読み込まれている
\usepackage{tikz-cd}
% \begin{tikzcd} hoge \end{tikzcd} により図式を書けるようにする
% \usepackage{tikz} + \usetikzlibrary{cd} でも同等の環境
% ※ 注意 : graphicx,xcolor のパッケージを読み込んだ後に読み込む必要がある
\usetikzlibrary{patterns,calc,arrows,decorations,angles,shapes.geometric,fit}
% patterns : 背景パターンを塗る 
% intersections : 交点を求める 
% calc : tikz の座標計算を可能にする
% arrows : 矢印の種類を増やす
% decorations : パスを飾る 
% angles : 角度記号を描く 
% shapes.geometric : いろいろな形を追加する
% fit : 複数座標を使える

%%%%%%%%%%%%%%%%%%%%%%%%%%%%%%%%%%%%%%%%%%%%%%%%%%%%%%%%%%%%%%%%%%%%%%%%%%%%%%%%%%%
%% 枠関係 %%
\usepackage{fancybox}
% shadowbox や doublebox 環境の枠の追加
\usepackage{ascmac}
% itembox 環境 やそのほか多数の枠の追加
\usepackage{tcolorbox} % https://www.ctan.org/pkg/tcolorbox
% ページまたぎが可能で tikz で枠を作成できるコマンド tcolorbox を使えるようにする
%%%%%%%%%%%%%%%%%%%%%%%%%%%%%%%%%%%%%%%%%%%%%%%%%%%%%%%%%%%%%%%%%%%%%%%%%%%%%%%%%%%
%% フォント関係 %%
\usepackage{calrsfs}
% mathcal などで calligraphic letters を出力できるようにする
\usepackage{mathrsfs}
% mathscr で花文字を出力できるようにする(花文字はいろいろ種類あるらしいので自分好みのパッケージを探してください)
\usepackage{amsfonts}
% 黒板太文字 mathbb や フラクトゥール mathfrak を使用できるようにする
\usepackage{bm}
% コマンド bm で太文字の斜体などを使えるようにする
%%%%%%%%%%%%%%%%%%%%%%%%%%%%%%%%%%%%%%%%%%%%%%%%%%%%%%%%%%%%%%%%%%%%%%%%%%%%%%%%%%%
%% その他 %%

\usepackage{enumerate}
% オプションを追加することで高性能なカウンターをコマンド enumerate で使えるようにする
\usepackage{paralist}
% 改行しない箇条書きのコマンド inparaenum を使えるようにする
\usepackage{here}
% 図の位置を指定するときに [H] をオプションに追加することで強制的にその場に出力する
\usepackage{geometry}
% 使う紙の大きさや、余白の長さなどを指定できる

%%%%%%%%%%%%%%%%%%%%%%%%%%%%%%%%%%%%%%%%%%%%%%%%%%%%%%%%%%%%%%%%%%%%%%%%%%%%%%%%%%%
%% hyperref 関係 %%
\usepackage{hyperref}% https://texwiki.texjp.org/?hyperref
% TeX 文書 (DVI、PDF など) に HTML と同じハイパーリンク 機能を加えるためのマクロを導入する
\usepackage{pxjahyper}
% hyperref などの日本語対応化パッケージ

\hypersetup{
  setpagesize=false,
  bookmarksdepth=section,
  % ブックマークをどの階層(section, subsection, subsubsection など)まで作成するか。
  bookmarksnumbered=true,
  % ブックマークに番号を付記するかどうか。
  colorlinks=true,
  % リンクに色を付けるかどうか。
  urlcolor=black,
  % 上が true のときにurlを何色にするか。
  citecolor=black,
  % \cite で参照したときの色をどうするか。
  linkcolor=black,  
  % \ref で参照したときの色をどうするか。
  pdftitle={black},
  pdfsubject={},
  pdfauthor={},
  pdfkeywords={}
  % pdf のメタ情報を定める。
}

\theoremstyle{definition}
\renewcommand{\proofname}{\textbf{証明}}
\newtheorem{thm}{Theorem}[section]
% 定理番号に section のカウンターを含める。
\newtheorem{lem}[thm]{Lemma}
\newtheorem{dfn}[thm]{Definition}
\newtheorem{cor}[thm]{Corollary}
\newtheorem{prop}[thm]{Proposition}
\newtheorem{ex}[thm]{Example}
\newtheorem{re}[thm]{Remark}
\newtheorem{exe}[thm]{Exercise}

\DeclareMathOperator{\Ob}{Ob}
\DeclareMathOperator{\Mor}{Mor}
\DeclareMathOperator{\Hom}{Hom}
\DeclareMathOperator{\ch}{ch}
\DeclareMathOperator{\rk}{rank}
\DeclareMathOperator{\GL}{GL}
\DeclareMathOperator{\Gr}{Gr}
\DeclareMathOperator{\Aut}{Aut}
\DeclareMathOperator{\End}{End}
\DeclareMathOperator{\Ext}{Ext}
\DeclareMathOperator{\Iso}{Iso}
\DeclareMathOperator{\Ker}{Ker}
\DeclareMathOperator{\rad}{rad}
\DeclareMathOperator{\Rep}{Rep}
\DeclareMathOperator{\Coh}{Coh}
\DeclareMathOperator{\Tor}{Tor}
\DeclareMathOperator{\Cl}{Cl}
\DeclareMathOperator{\Supp}{Supp}
\DeclareMathOperator{\UFD}{UFD}
\DeclareMathOperator{\PID}{PID}
\DeclareMathOperator{\im}{Im}
\DeclareMathOperator{\Coker}{Coker}
\DeclareMathOperator{\Frac}{Frac}
\DeclareMathOperator{\Gal}{Gal}
\DeclareMathOperator{\Frob}{Frob}
%%%%%%%%%%%%%%%%%%%%%%%%%%%%%%%%%%%%%%%%%%%%%%%%%%%%%%%%%%%%%%%%%%%%%%%%%%%%

% \numberwithin{equation}{subsection}
% ※ 注意 : % を外して使用する
% 数式番号に subsection のカウンターを含める。
%%%%%%%%%%%%%%%%%%%%%%%%%%%%%%%%%%%%%%%%%%%%%%%%%%%%%%%%%%%%%%%%%%%%%%%%%%%%