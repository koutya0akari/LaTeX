% 通常 platex + dvipdfmx で動作している
% (u)platex は DVI 出力エンジン, 対して pdfTeX, LuaTeX, XeTeX PDF 出力エンジン
% DVI 出力エンジンを使用する場合, ドライバ指定をする必要がある
\documentclass[dvipdfmx]{bxjsreport}
% \documentclass[dvipdfmx]{jsarticle}
% \documentclass[dvipdfmx]{bxjsbook}
% dvipdfmx でドライバ指定している理由は graphicx,xcolor などがドライバ依存のパッケージであるため
%%%%%%%%%%%%%%%%%%%%%%%%%%%%%%%%%%%%%%%%%%%%%%%%%%%%%%%%%%%%%%%%%%%%%%%%%%%%%%%%%%%
%% \usepackage{graphicx}
% png や jpeg といった図を pdf に挿入できるようにする
% ※ 注意 : tikzより前に読み込む必要あり
\usepackage[x11names]{xcolor} % マニュアル : コマンドラインから texdoc xcolor を実行
% xoclor : 文字などの書式に色をつけることができる
% x11names : 使える色名が317色(rgb)増える. デフォルトでは 19色
% デフォルト19色 : red,green,blue,cyan,magenta,yellow,black,gray,white,darkgray,lightgray,brown,lime,olive,orange,pink,purple,teal,violet
% ※ 注意 : graphicx,xcolor はどちらも dvipdfmx のドライバ指定が必須
% ※ 注意 : オプションを使用する場合, tikzより前に読み込む必要あり
%%%%%%%%%%%%%%%%%%%%%%%%%%%%%%%%%%%%%%%%%%%%%%%%%%%%%%%%%%%%%%%%%%%%%%%%%%%%%%%%%%%
%% 基本 %%

\usepackage{amsmath,mathtools}
% 数式関数 align などを使えるようにする
\usepackage{amsthm}
% 定理環境 theoremstyle 制作
\usepackage{amssymb}
% 基本的な数学記号などの導入
\usepackage{latexsym}
% 数式で使える記号を増やす
%%%%%%%%%%%%%%%%%%%%%%%%%%%%%%%%%%%%%%%%%%%%%%%%%%%%%%%%%%%%%%%%%%%%%%%%%%%%%%%%%%%
%% tikz (てぃくす)関係 %% % https://texwiki.texjp.org/?TikZ

\usepackage{tikz}
% \begin{tikzpicture} hoge \end{tikzpicture} により図式を書けるようにする
% beamer の基礎エンジンでもある
% ※ 注意 : graphicx,xcolor 読み込まれている
\usepackage{tikz-cd}
% \begin{tikzcd} hoge \end{tikzcd} により図式を書けるようにする
% \usepackage{tikz} + \usetikzlibrary{cd} でも同等の環境
% ※ 注意 : graphicx,xcolor のパッケージを読み込んだ後に読み込む必要がある

\usetikzlibrary{patterns,calc,arrows,decorations,angles,shapes.geometric,fit}
% patterns : 背景パターンを塗る 
% intersections : 交点を求める 
% calc : tikz の座標計算を可能にする
% arrows : 矢印の種類を増やす
% decorations : パスを飾る 
% angles : 角度記号を描く 
% shapes.geometric : いろいろな形を追加する
% fit : 複数座標を使える

%%%%%%%%%%%%%%%%%%%%%%%%%%%%%%%%%%%%%%%%%%%%%%%%%%%%%%%%%%%%%%%%%%%%%%%%%%%%%%%%%%%
\usepackage{fancybox}
% shadowbox や doublebox 環境の枠の追加
\usepackage{ascmac}
% itembox 環境 やそのほか多数の枠の追加
\usepackage{tcolorbox} % https://www.ctan.org/pkg/tcolorbox
% ページまたぎが可能で tikz で枠を作成できるコマンド tcolorbox を使えるようにする
%%%%%%%%%%%%%%%%%%%%%%%%%%%%%%%%%%%%%%%%%%%%%%%%%%%%%%%%%%%%%%%%%%%%%%%%%%%%%%%%%%%
\usepackage{multicol}
% 段組の書式などを出力できるようにする
%%%%%%%%%%%%%%%%%%%%%%%%%%%%%%%%%%%%%%%%%%%%%%%%%%%%%%%%%%%%%%%%%%%%%%%%%%%%%%%%%%%
\usepackage{listings,jvlisting}
% 日本語のコメントアウトをする場合 jvlisting (もしくは jlisting ) が必要
%%%%%%%%%%%%%%%%%%%%%%%%%%%%%%%%%%%%%%%%%%%%%%%%%%%%%%%%%%%%%%%%%%%%%%%%%%%%%%%%%%%
\usepackage{calrsfs}
% mathcal などで calligraphic letters を出力できるようにする
\usepackage{mathrsfs}
% \mathscr で花文字を出力できるようにする(花文字はいろいろ種類あるらしいので自分好みのパッケージを探してください)
\usepackage{amsfonts}
% 黒板太文字 mathbb や フラクトゥール mathfrak を使用できるようにする
\usepackage{bm}
% コマンド bm で太文字の斜体などを使えるようにする
%%%%%%%%%%%%%%%%%%%%%%%%%%%%%%%%%%%%%%%%%%%%%%%%%%%%%%%%%%%%%%%%%%%%%%%%%%%%%%%%%%%

\usepackage{wrapfig}
% 図を回り込ませるためのパッケージ
\usepackage{MnSymbol}
\usepackage{varwidth} 
\usepackage{framed,color}
% 改ページ可能なフレームの出力
\usepackage{enumerate}
% オプションを追加することで高性能なカウンターをコマンド enumerate で使えるようにする
\usepackage{paralist}
% 改行しない箇条書きのコマンド, inparaenum を使えるようにする
\usepackage{here}
% 図の位置を指定するときに [H] をオプションに追加することで強制的にその場に出力する
\usepackage{setspace}
% 行間を変更できるようにする
\usepackage{geometry}
% 使う紙の大きさや、余白の長さなどを指定できる
\usepackage{pst-solides3d}
\usepackage{pgfplots}

\usepackage{fancyhdr}
% ページ番号を左上に表示する
%%%%%%%%%%%%%%%%%%%%%%%%%%%%%%%%%%%%%%%%%%%%%%%%%%%%%%%%%%%%%%%%%%%%%%%%%%%%%%%%%%%
%% hyperref 関係 %%
\usepackage{hyperref}
% TeX 文書 (DVI、PDF など) に HTML と同じハイパーリンク 機能を加えるためのマクロを導入する
% https://texwiki.texjp.org/?hyperref
\usepackage{pxjahyper}
% hyperref などの日本語対応化パッケージ
\usepackage[nameinlink]{cleveref}
% 高性能な hyperref 
% nameinlink : 参照の部分全体をハイパーリンクにできる
% ※ 注意 : hyperref の後に導入する必要がある。
\usepackage{aliascnt}
% ref による参照を autoref により自動化する
%%%%%%%%%%%%%%%%%%%%%%%%%%%%%%%%%%%%%%%%%%%%%%%%%%%%%%%%%%%%%%%%%%%%%%%%%%%%%%%%%%%
\tcbuselibrary{many}
%\geometry{top=20mm, bottom=20mm, left=25mm, right=25mm}
\tikzset{cross/.style={preaction={-,draw=white,line width=6pt}}}
%%%%%%%%%%%%%%%%%%%%%%%%%%%%%%%%%%%%%%%%%%%%%%%%%%%%%%%%%%%%%%%%%%%%%%%%%%%%
\definecolor{pop}{RGB}{106,90,205}
\hypersetup{
  setpagesize=false,
  bookmarks=true,
  bookmarksdepth=tocdepth,
  bookmarksnumbered=true,
  colorlinks=true,
  urlcolor=cyan,
  citecolor=pop,
  linkcolor=pop,  
  pdftitle={cyan},
  pdfsubject={},
  pdfauthor={},
  pdfkeywords={}
}

\lstset{
  basicstyle={\ttfamily},
  identifierstyle={\small},
  commentstyle={\smallitshape},
  keywordstyle={\small\bfseries},
  ndkeywordstyle={\small},
  stringstyle={\small\ttfamily},
  frame={tb},
  breaklines=true,
  columns=[l]{fullflexible},
  numbers=left,
  xrightmargin=0zw,
  xleftmargin=3zw,
  numberstyle={\scriptsize},
  stepnumber=1,
  numbersep=1zw,
  lineskip=-0.5ex
}

%%%%%%%%%%%%%%%%%%%%%%%%%%%%%%%%%%%%%%%%%%%%%%%%%%%%%%%%%%%%%%%%%%%%%%%%%%%%
%% 現在必要ないパッケージ 群%%
%\usepackage[xcolor]{mdframed}
% 枠の作成に TikZ を利用することができる
%\usepackage{silence}\WarningFilter{mdframed}{You got a bad break}
%%%%%%%%%%%%%%%%%%%%%%%%%%%%%%%%%%%%%%%%%%%%%%%%%%%%%%%%%%%%%%%%%%%%%%%%%%%%

\newtheorem{mistake}{誤植}
\newtheorem{fix}{修正}

\tcolorboxenvironment{mistake}
{enhanced, colback=white, colframe=white,
fonttitle=\bfseries, coltitle=black, fonttitle=\gtfamily, 
breakable, sharp corners,
boxed title style={colback=white,left=0pt,right=0pt}, 
top=2pt,bottom=2pt,left=2pt,right=2pt,
borderline west={.75pt}{2pt}{black,dotted},
underlay unbroken={
  \filldraw[fill=gray,draw=gray]
  (frame.south west) -- +(0pt,5pt) -- +(5pt,5pt) -- +(5pt,0pt) -- cycle;
  \filldraw[fill=gray,draw=gray]
  (frame.north west) -- +(0pt,-5pt) -- +(5pt,-5pt) -- +(5pt,0pt) -- cycle;
  },
underlay first={
  \filldraw[fill=gray,draw=gray]
  (frame.north west) -- +(0pt,-5pt) -- +(5pt,-5pt) -- +(5pt,0pt) -- cycle;
  },
underlay middle={
  \draw[black,line width=.5pt]
  (frame.south west)--(frame.north west) ;
  },
underlay last={
  \filldraw[fill=gray,draw=gray]
  (frame.south west) -- +(0pt,5pt) -- +(5pt,5pt) -- +(5pt,0pt) -- cycle;
  }
}

\tcolorboxenvironment{fix}
{enhanced, colback=white, colframe=white,
fonttitle=\bfseries, coltitle=black, fonttitle=\gtfamily, 
breakable, sharp corners,
boxed title style={colback=white,left=0pt,right=0pt}, 
top=2pt,bottom=2pt,left=2pt,right=2pt,
borderline west={.75pt}{2pt}{black},
underlay unbroken={
  \filldraw[fill=gray,draw=gray]
  (frame.south west) -- +(0pt,5pt) -- +(5pt,5pt) -- +(5pt,0pt) -- cycle;
  \filldraw[fill=gray,draw=gray]
  (frame.north west) -- +(0pt,-5pt) -- +(5pt,-5pt) -- +(5pt,0pt) -- cycle;
  },
underlay first={
  \filldraw[fill=gray,draw=gray]
  (frame.north west) -- +(0pt,-5pt) -- +(5pt,-5pt) -- +(5pt,0pt) -- cycle;
  },
underlay middle={
  \draw[black,line width=.5pt]
  (frame.south west)--(frame.north west) ;
  },
underlay last={
  \filldraw[fill=gray,draw=gray]
  (frame.south west) -- +(0pt,5pt) -- +(5pt,5pt) -- +(5pt,0pt) -- cycle;
  }
}

\newtcolorbox{lbar}[1]
{enhanced, colback=white, colframe=white,
fonttitle=\bfseries, coltitle=black, fonttitle=\gtfamily, 
breakable, sharp corners,
boxed title style={colback=white,left=0pt,right=0pt}, 
top=2pt,bottom=2pt,left=2pt,right=2pt,
borderline west={1pt}{2pt}{black},
title={#1}
}

\newtheoremstyle{mydefinition}{\topsep}{\topsep}{\rmfamily}{0pt}{\sffamily\gtfamily\bfseries}{\newline}{1em}
{\underline{\thmname{#1}\thmnumber{ #2}\thmnote{ : #3 }\rmfamily}}

\theoremstyle{mydefinition}


% \newaliastheorem{(環境とカウンターの名前)}{(元となるカウンターの名前)}{(表示される文字列)}
%\newcommand*{\newaliastheorem}[3]{%
%  \newaliascnt{#1}{#2}%
%  \newtheorem{#1}[#1]{#3}%
%  \aliascntresetthe{#1}%
%  \expandafter\newcommand\csname #1autorefname\endcsname{#3}%
%}

\newtheorem{thm}{Theorem}[section]
\crefname{thm}{Theorem}{Theorems}
\newtheorem{lem}[thm]{Lemma}
\crefname{lem}{Lemma}{Lemmas}
\newtheorem{dfn}[thm]{Definition}
\crefname{dfn}{Definition}{Definitions}
\newtheorem{cor}[thm]{Corollary}
\crefname{cor}{Corollary}{Corollarys}
\newtheorem{prop}[thm]{Proposition}
\crefname{prop}{Proposition}{Propositions}
\newtheorem{ex}[thm]{Example}
\crefname{ex}{Example}{Examples}
\newtheorem{re}[thm]{Remark}
\crefname{re}{Remark}{Remarks}
\newtheorem{exe}[thm]{Exercise}
\crefname{exe}{Exercise}{Exercises}
%\renewcommand{\theoremautorefname}{定理}
%\newaliastheorem{lem}{thm}{Lemma}
%\newaliastheorem{dfn}{thm}{Definition}
%\newaliastheorem{cor}{thm}{Corollary}
%\newaliastheorem{prop}{thm}{Proposition}
%\newaliastheorem{ex}{thm}{Example}
%\newaliastheorem{re}{thm}{Remark}
%\newaliastheorem{exe}{thm}{Exercise}

\tcolorboxenvironment{thm}
{enhanced, colback=white, colframe=white,
fonttitle=\bfseries, coltitle=black, fonttitle=\gtfamily, 
breakable, sharp corners,
boxed title style={colback=white,left=0pt,right=0pt}, 
top=4pt,bottom=4pt,left=4pt,right=4pt,
borderline west={0.75pt}{0.75pt}{black},
}

\tcolorboxenvironment{lem}
{enhanced, colback=white, colframe=white,
fonttitle=\bfseries, coltitle=black, fonttitle=\gtfamily, 
breakable, sharp corners,
boxed title style={colback=white,left=0pt,right=0pt}, 
top=4pt,bottom=4pt,left=4pt,right=4pt,
borderline west={0.75pt}{0.75pt}{black},
}

\tcolorboxenvironment{dfn}
{enhanced, colback=white, colframe=white,
fonttitle=\bfseries, coltitle=black, fonttitle=\gtfamily, 
breakable, sharp corners,
boxed title style={colback=white,left=0pt,right=0pt}, 
top=4pt,bottom=4pt,left=4pt,right=4pt,
borderline west={0.75pt}{0.75pt}{black},
}

\tcolorboxenvironment{cor}
{enhanced, colback=white, colframe=white,
fonttitle=\bfseries, coltitle=black, fonttitle=\gtfamily, 
breakable, sharp corners,
boxed title style={colback=white,left=0pt,right=0pt}, 
top=4pt,bottom=4pt,left=4pt,right=4pt,
borderline west={0.75pt}{0.75pt}{black},
}

\tcolorboxenvironment{prop}
{enhanced, colback=white, colframe=white,
fonttitle=\bfseries, coltitle=black, fonttitle=\gtfamily, 
breakable, sharp corners,
boxed title style={colback=white,left=0pt,right=0pt}, 
top=4pt,bottom=4pt,left=4pt,right=4pt,
borderline west={0.75pt}{0.75pt}{black},
}

\tcolorboxenvironment{ex}
{enhanced, colback=white, colframe=white,
fonttitle=\bfseries, coltitle=black, fonttitle=\gtfamily, 
breakable, sharp corners,
boxed title style={colback=white,left=0pt,right=0pt}, 
top=4pt,bottom=4pt,left=4pt,right=4pt,
borderline west={0.75pt}{0.75pt}{black},
}

\tcolorboxenvironment{re}{
  enhanced,
  sharp corners,
  breakable,
  colback=white, colframe=white,
  top=4pt,bottom=4pt,left=4pt,right=4pt,
  boxrule=0.75pt,
  borderline={0.75pt}{0.75pt}{black},
}

\tcolorboxenvironment{exe}
{enhanced, colback=white, colframe=white,
fonttitle=\bfseries, coltitle=black, fonttitle=\gtfamily, 
breakable, sharp corners,
boxed title style={colback=white,left=0pt,right=0pt}, 
top=4pt,bottom=4pt,left=4pt,right=4pt,
borderline west={0.75pt}{0.75pt}{black},
}

\makeatletter % use at mark
\renewenvironment{proof}[1][\proofname]{\par
  \pushQED{\qed}%
  \normalfont \topsep6\p@\@plus6\p@\relax
  \trivlist
  \item[\hskip\labelsep
        \itshape
    {\bf{\underline{#1. }}}]\ignorespaces
    % {\bf\underline{#1}\@addpunct{.}}]\ignorespaces % ピリオドあり
}{%
  \popQED\endtrivlist\@endpefalse
}
\makeatother % end at mark
%%%%%%%%%%%%%%%%%%%%%%%%%%%%%%%%%%%%%%%%%%%%%%%%%%%%%%%%%%%%%%%%%%%%%%%%%%%%
\newcommand{\resp}[1]{($\mathbf{resp}$. #1)}
\newcommand{\Rem}[1]{($\mathbf{Remark}$. #1)}
\renewcommand\thefootnote{*\arabic{footnote}}
%%%%%%%%%%%%%%%%%%%%%%%%%%%%%%%%%%%%%%%%%%%%%%%%%%%%%%%%%%%%%%%%%%%%%%%%%%%%
%マクロ
%%%%%%%%%%%%%%%%%%%%%%%%%%%%%%%%%%%%%%%%%%%%%%%%%%%%%%%%%%%%%%%%%%%%%%%%%%%%
\newcommand{\A}{\mathbb{A}}
\newcommand{\B}{\mathbb{B}}
\newcommand{\C}{\mathbb{C}}
\newcommand{\D}{\mathbb{D}}
\newcommand{\E}{\mathbb{E}}
\newcommand{\F}{\mathbb{F}}
\newcommand{\G}{\mathbb{G}}
\newcommand{\I}{\mathbb{I}}
\newcommand{\J}{\mathbb{J}}
\newcommand{\K}{\mathbb{K}}
\newcommand{\M}{\mathbb{M}}
\newcommand{\N}{\mathbb{N}}
\renewcommand{\O}{\mathbb{O}}
\renewcommand{\P}{\mathbb{P}}
\newcommand{\Q}{\mathbb{Q}}
\newcommand{\R}{\mathbb{R}}
\newcommand{\T}{\mathbb{T}}
\newcommand{\U}{\mathbb{U}}
\newcommand{\V}{\mathbb{V}}
\newcommand{\W}{\mathbb{W}}
\newcommand{\X}{\mathbb{X}}
\newcommand{\Y}{\mathbb{Y}}
\newcommand{\Z}{\mathbb{Z}}
\renewcommand{\a}{\mathbb{a}}
\renewcommand{\b}{\mathbb{b}}
\renewcommand{\c}{\mathbb{c}}
\renewcommand{\d}{\mathbb{d}}
\newcommand{\e}{\mathbb{e}}
\newcommand{\f}{\mathbb{f}}
\newcommand{\g}{\mathbb{g}}
\renewcommand{\i}{\mathbb{i}}
\renewcommand{\j}{\mathbb{j}}
\renewcommand{\k}{\mathbb{k}}
\newcommand{\m}{\mathbb{m}}
\renewcommand{\o}{\mathbb{o}}
\newcommand{\p}{\mathbb{p}}
\newcommand{\q}{\mathbb{q}}
\renewcommand{\r}{\mathbb{r}}
\renewcommand{\t}{\mathbb{t}}
\renewcommand{\u}{\mathbb{u}}
\renewcommand{\v}{\mathbb{v}}
\newcommand{\w}{\mathbb{w}}
\newcommand{\x}{\mathbb{x}}
\newcommand{\y}{\mathbb{y}}
\newcommand{\z}{\mathbb{z}}
%%%%%%%%%%%%%%%%%%%%%%%%%%%%%%%%%%%%%%%%%%%%%%%%%%%%%%%%%%%%%%%%%%%%%%%%%%%%
\newcommand{\liea}{\mathfrak{a}}
\newcommand{\lieb}{\mathfrak{b}}
\newcommand{\liec}{\mathfrak{c}}
\newcommand{\lied}{\mathfrak{d}}
\newcommand{\liee}{\mathfrak{e}}
\newcommand{\lief}{\mathfrak{f}}
\newcommand{\lieg}{\mathfrak{g}}
\newcommand{\lieh}{\mathfrak{h}}
\newcommand{\liei}{\mathfrak{i}}
\newcommand{\liej}{\mathfrak{j}}
\newcommand{\liek}{\mathfrak{k}}
\newcommand{\liel}{\mathfrak{l}}
\newcommand{\liem}{\mathfrak{m}}
\newcommand{\lien}{\mathfrak{n}}
\newcommand{\lieo}{\mathfrak{o}}
\newcommand{\liep}{\mathfrak{p}}
\newcommand{\lieq}{\mathfrak{q}}
\newcommand{\lier}{\mathfrak{r}}
\newcommand{\lies}{\mathfrak{s}}
\newcommand{\liet}{\mathfrak{t}}
\newcommand{\lieu}{\mathfrak{u}}
\newcommand{\liev}{\mathfrak{v}}
\newcommand{\liew}{\mathfrak{w}}
\newcommand{\liex}{\mathfrak{x}}
\newcommand{\liey}{\mathfrak{y}}
\newcommand{\liez}{\mathfrak{z}}
%%%%%%%%%%%%%%%%%%%%%%%%%%%%%%%%%%%%%%%%%%%%%%%%%%%%%%%%%%%%%%%%%%%%%%%%%%%%
\newcommand{\vca}{\bm{a}}
\newcommand{\vcb}{\bm{b}}
\newcommand{\vcc}{\bm{c}}
\newcommand{\vcd}{\bm{d}}
\newcommand{\vce}{\bm{e}}
\newcommand{\vcf}{\bm{f}}
\newcommand{\vcg}{\bm{g}}
\newcommand{\vch}{\bm{h}}
\newcommand{\vci}{\bm{i}}
\newcommand{\vcj}{\bm{j}}
\newcommand{\vck}{\bm{k}}
\newcommand{\vcl}{\bm{l}}
\newcommand{\vcm}{\bm{m}}
\newcommand{\vcn}{\bm{n}}
\newcommand{\vco}{\bm{o}}
\newcommand{\vcp}{\bm{p}}
\newcommand{\vcq}{\bm{q}}
\newcommand{\vcr}{\bm{r}}
\newcommand{\vcs}{\bm{s}}
\newcommand{\vct}{\bm{t}}
\newcommand{\vcu}{\bm{u}}
\newcommand{\vcv}{\bm{v}}
\newcommand{\vcw}{\bm{w}}
\newcommand{\vcx}{\bm{x}}
\newcommand{\vcy}{\bm{y}}
\newcommand{\vcz}{\bm{z}}
%%%%%%%%%%%%%%%%%%%%%%%%%%%%%%%%%%%%%%%%%%%%%%%%%%%%%%%%%%%%%%%%%%%%%%%%%%%%
\newcommand{\Ra}{\Rightarrow}
\newcommand{\Le}{\Leftarrow}
%%%%%%%%%%%%%%%%%%%%%%%%%%%%%%%%%%%%%%%%%%%%%%%%%%%%%%%%%%%%%%%%%%%%%%%%%%%%
%%%%%%%%%%%%%%category%%%%%%%%%%%%%%%
\newcommand{\CC}{\mathcal{C}}
\newcommand{\DD}{\mathcal{D}}
\newcommand{\id}{\mathrm{id}}
\newcommand{\Set}{\textbf{Set}}
\newcommand{\Grp}{\textbf{Grp}}
\newcommand{\Cat}{\textbf{Cat}}
\newcommand{\Top}{\textbf{Top}}
\DeclareMathOperator{\Ob}{Ob}
\DeclareMathOperator{\Mor}{Mor}
\DeclareMathOperator{\Hom}{Hom}
%%%%%%%%%%%%%%%%%%%%%%%%%%%%%%%%%%%%%%%%%%%%%%%%%%%%%%%%%%%%%%%%%%%%%%%%%%%%
\DeclareMathOperator{\ch}{ch}
\DeclareMathOperator{\rk}{rank}
\DeclareMathOperator{\GL}{GL}
\DeclareMathOperator{\Gr}{Gr}
\DeclareMathOperator{\Aut}{Aut}
\DeclareMathOperator{\End}{End}
\DeclareMathOperator{\Ext}{Ext}
\DeclareMathOperator{\Iso}{Iso}
\DeclareMathOperator{\Ker}{Ker}
\DeclareMathOperator{\rad}{rad}
\DeclareMathOperator{\Rep}{Rep}
\DeclareMathOperator{\Coh}{Coh}
\DeclareMathOperator{\Tor}{Tor}
\DeclareMathOperator{\Cl}{Cl}
\DeclareMathOperator{\Supp}{Supp}
\DeclareMathOperator{\UFD}{UFD}
\DeclareMathOperator{\PID}{PID}
\DeclareMathOperator{\im}{Im}
\DeclareMathOperator{\Coker}{Coker}
\DeclareMathOperator{\Frac}{Frac}
\DeclareMathOperator{\Gal}{Gal}
\DeclareMathOperator{\Frob}{Frob}
%%%%%%%%%%%%%%%%%%%%%%%%%%%%%%%%%%%%%%%%%%%%%%%%%%%%%%%%%%%%%%%%%%%%%%%%%%%%

\numberwithin{equation}{subsection}
\makeatletter
\@addtoreset{equation}{subsection}
\@addtoreset{figure}{section}
\@addtoreset{table}{section}

%%%%%%%%%%%%%%%%%%%%%%%%%%%%%%%%%%%%%%%%%%%%%%%%%%%%%%%%%%%%%%%%%%%%%%%%%%%%


%%%%%%%%%%%%%%%%%%%%%%%%%%%%%%%%%%%%%%%%%%%%%%%%%%%%%%%%%%%%%%%%%%%%%%%%%%%%